
\chapterimage{header.pdf} % Chapter heading image

\chapter{Introdução}

Todas as vezes que eu leio um livro a parte que eu mais detesto é a introdução, então eu recomendo fortemente que você, meu amigo 
leitor, vá direto ao próximo capítulo. Porém, entretando, todavia, se você for um fraco, assim como eu, e gostar de uma introduzida 
(tcheeee), farei esse agrado para você. Mas antes de mais nada eu gostaria de pedir desculpas pelo conteúdo e linguajar de baixo 
escalão que será utilizado nesse livro. Não sei ainda o que é pior, se são as besteiras, o português ou a linguagem C. As besteiras 
expressam melhor o meu sentimento (ERM), a linguagem C, porra, essa temos que respeitar, ela é digna de dar o nome do seu filho de 
\textit{C ANSI}, e o português aprendi com um ano de idade e até hoje eu me considero um \textit{noob}.


\section{Como contribuir?}

É muito fácil contribuir com um projeto de merda, qualquer programador .NET consegue! (brincadeirinha galera!). 
Deixarei algumas dicas de como ajudar.

\begin{itemize}
 \item Traduzindo este documento para outras linguas;
 \item Escrevendo algum capítulo técnico interessante;
 \item Corrigindo bugs;
 \item Com idéias;
 \item Corrigindo erros de português (RÁÁÁ pegadinha do malandro);
\end{itemize}

\section{Quem deveria ler este livro?}

Sinceramente falando acho que ninguém, mas se você é um gerente de projeto, com tempo livre (nunca vi gerente de projeto ocupado), 
saia do facebook e venha conhecer um pouco do submundo dos programadores de merda. Além dos gerentes de projetos quem mais pode ler:

\begin{itemize}
  \item Pessoas com tempo livre;
  \item Péssimos programadores;
  \item Programadores java;
  \item Pessoas que acreditam que seu código é seu filho;
\end{itemize}

\section{Quem não deveria ler este livro?}

Pessoas que se ofendem facilmente, idosos, cardíacos, gestantes, menores de 18 anos e bons programadores.

\section{Como este livro funciona?}

A idéia desse livro não é ter apenas um ou dois autores, mas vários, qualquer um está livre para colaborar. Cada capítulo do livro 
é indepentente, ou seja, você poderá encontrar um capítulo sobre um tópico avançado sem ter um capítulo introdutório anterior. 
Então agora é só baixar o projeto e começar a escrever seu capítulo, em latex.

\begin{verbatim}
 $ git clone https://github.com/patito/DiarioDeUmProgramadorDeMerda.git
\end{verbatim}




